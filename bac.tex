\documentclass[thesismargins, thesislinespacing, twoside, upjsfrontpage]{rnthesis}
\usepackage[slovak]{babel}
\usepackage[utf8]{inputenc}
\usepackage{graphicx}

\title{Získavanie priestorových obrazov}
\author{Matej Nikorovič}
\typprace{bakalárska}
\rok{2012}
\miesto{Košice}
\podakovanie{Moja vďaka patrí môjmu vedúcemu bakalárskej práce Mgr. Ladislavovi Mikešovi za pomoc a vedenie tejto práce.}

\veduci{RNDr. Ladislav Mikeš}
\pracovisko{Ústav informatiky}
\studprog{Ib - Informatika}
\odbor{9.2.1. - Informatika}

\abstrakt{V tejto praci sa budeme zaoberať získavaním priestorových obrazov, čo je jeden z objektov záujmu \emph{počítačového videnia}. Jej prvá časť popisuje a porovnáva vybrané metódy získavania 3D dát a zvyšok práce sa venuje návrhu a zostrojením nástroja na 3D skenovanie pomocou \emph{štruktúrovaného svetla}.}

\abstract{abstract EN}

\begin{document}
\maketitle
\newpage
\tableofcontents
\listoffigures
\listoftables

\chapter*{Úvod}
\addcontentsline{toc}{chapter}{Úvod}

\chapter{Metódy získavania 3D dát}

\paragraph{3D model} je matematická reprezentácia povrchu priestorového telesa. 3D modely môžme rozdeliť do dvoch skupín:
\begin{itemize}
	\item Hraničná reprezentácia - definuje hranicu telesa. %Našli si použitie vo vizuálnych
	\begin{enumerate}
		\item Mračno bodov (Point cloud) - množina bodov telesa.
		\item Drôtený model (Wire frame) - množina bodov telesa a ich spájajúcich hrán.
		\item Hraničné steny (Polygon mesh) - množina stien telesa.
		%\item Implicitné zadanie
		%\item Parametrické zadanie
	\end{enumerate}
	\item Objemová reprezentácia - definuje objem, ktorý teleso zaberá. %Tieto modely sú realistickejšie
	\begin{enumerate}
		\item Bunkový model - skladá sa z voxelov (voxel = volume element) tj. či sa na danom mieste nachádza / nenachádza teleso.
		\item Oktantový strom (Octree) - stromová štruktúra bunkového modelu vytvorená kvôli šetreniu pamäte.
		\item CGS reprezentácia (Constructive solid geometry) - teleso je definované CSG stromom, ktorý pozostáva z množiny operácií, transformácií a primitív (základných geometrických telies).
	\end{enumerate}
\end{itemize}

\section{3D skener}
3D skener je zariadenie na získavanie digitálnych informácií o tvare telesa reálneho sveta. Výsledným produktom 3D skenovania je 3D model, ktorý je zvyčajne len mračno bodov. Vo väčšine prípadov 1 snímok na vytvorenie 3D modelu nestačí a preto sa sníma na viackrát a z rôznych uhlov pohľadu. Princíp skenera, akým získava hĺbkovú informáciu, je závislý od použitej technológie. Každá z týchto technológií má svoje silné aj slabé stránky, svoju cenu alebo fyzikálne či matematické obmedzenia. Potom podľa použitej technológie môžme skenery rozdeliť na dva veľké celky: kontaktné a bezkontaktné 3D skenery.
\subsection{Kontaktné 3D skenery}
Kontaktné 3D skenery využívajú fyzický dotyk s telesom prichyteným k podložke. Snímanie dotykom je tradičná metóda na digitalizáciu telies v priemysle. Digitálne informácie získavame z dotykového senzora, ktorý môže byť zautomatizovaný alebo rukou ovládaný. Zariadenie, ktoré používa zautomatizovaný senzor na snímanie, nazývame CMM (coordinate measuring machine) a používa sa na overovanie rozmerov vyrobených dielov. Senzory ovládané rukou vyzerajú podobne ako pero. Ak užívateľ sa ním dotýka o daný povrch, tak sú schopné určiť priestorový bod v danom čase.

Kontaktné skenery sú veľmi presné (až na niekoľko mikrometrov). Dokážu snímať telesá s priehľadným, laským či zrakadlovým povrchom. Na druhej strane fyzický kontakt sa nedajú snímať krehké alebo mekké telesá. Zároveň merajú iba jeden bod na telese a tým pádom nedokážu dosiahnúť vyššie rýchlosti snímania.
\subsection{Bezkontaktné 3D skenery}
\subsubsection{Aktívne bezkontaktné 3D skenery}
\paragraph{Time-of-flight 3D skener}
\paragraph{Triangulácia}
\paragraph{3D skener využívajúci modulované svetlo}
\paragraph{3D skener využívajúci štruktúrované svetlo}
\subsubsection{Pasívne bezkontaktné 3D skenery}
\paragraph{Stereoskopické systémy}
\paragraph{Fotometrika}

\chapter{Princípy získavania 3D dát}
\section{Segmentácia}
\section{Binárna matematická morfológia}
\subsection{Dilatácia}
\subsection{Erózia}
\subsection{Otvorenie}
\subsection{Uzavretie}

\chapter{Súčasný stav problematiky}
3D skenovanie pomocou štruktúrovaného sveta je veľmi aktívna oblasť výskumu. Každým rokom pribúda počet vedeckých článkov v tejto oblasti. K najznámejším 3D skenerom využívajúcich štruktúrované svetlo je Microsoft Kinect. Známy je aj David Laserscanner.
% alebo automatické projektory so vstavanými svetelnými senzormi.
\section{Kinect}
\section{David Laserscanner}
David Laserscanner je softvérový balíček určený na lacné 3D skenovanie. Umožňuje nám získať 3D model reálnch telies s použitím kamery (stačí aj webkamera), rukou držaného čiarového lasera a dvoch na seba kolmých sten umiestnených v pozadí. Jeho najvýznačnejšia črta je snímanie časti telesa čiarovým laserom, ktoré ešte neboli nasnímané, keďže človek ovláda rukou čiarový laser. Snímanie telesa beží v reálnom čase a aktuálne výsledky sú zobrazované na obrazovke. Aplikácia dokáže exportovať výsledný 3D model do známych súborových formátov. Balíček obsahuje aj nástroj na ofarbenie tohto 3D modelu a nástroj na zlepenie modelov, získaných skenovaním rovnakého telesa z rôznych uhlov pohľadu.
\section{Automaticky kalibrovateľné projektory}


\chapter{Návrh a implemetácia 3D skenera využívajúceho štruktúrované svetlo}

\chapter{Výsledky práce}

\chapter{Záver}

\begin{thebibliography}{1}
\bibitem{lit0}BRADSKI G., KAEHLER A.: Learning OpenCV: Computer Vision with the OpenCV Library. 1. vydanie. O'Reilly Media, 2008. ISBN 0596516134.
\bibitem{lit1}CYGANEK B., SIEBERT J. P.: An Introduction to 3D Computer Vision Techniques and Algorithms. Wiley, 2009. ISBN 9780470017043.
\bibitem{lit2}FORSYTH D., PONCE J.: Computer Vision: A Modern Approach. 1. vydanie. Prentice Hall, 2002. ISBN 0130851981.
\bibitem{lit3}SZELISKI R.: Computer Vision: Algorithms and Applications. 1. vydanie. Springer, 2010. ISBN 9781848829343.
\bibitem{lit4}TRUCCO E., VERRI A.: Introductory Techniques for 3-D Computer Vision. Prentice Hall, 1998. ISBN 0132611082
\bibitem{lit5}ZLATOŠ P.: Lineárna algebra a geometria.  Albert Marenčin - Vydavateľstvo PT, 2011. ISBN 9788081141119
\bibitem{lit6}SOULIÉ, Juan.: C++ Language Tutorial. Cplusplus.com: Jún 2007. [cit. 21.8.2011] Dostupný na internete [http://www.cplusplus.com/doc/tutorial/].

\end{thebibliography}

\end{document}