\documentclass[thesismargins, thesislinespacing, thesischapterstyle, twoside, openright, upjsfrontpage]{rnthesis}
\usepackage[slovak]{babel}
\usepackage[T1]{fontenc}
\usepackage[utf8]{inputenc}
\usepackage{lmodern}

\usepackage{rnt-pic}
\usepackage{rnt-thm}

\usepackage{pdfpages} %vkladanie pdf stran do prace

\usepackage[hyphens]{url} % format a zalamovanie URL adries

\title{Umenie nebyť videný}
\author{Bc. František Tarabák}
\typprace{diplomová}
\rok{2012}
\miesto{Košice}
\podakovanie{
  Rád by som poďakoval vedúcemu diplomovej
  práce akad. Františkovi Tarabákovi, DrSc.
  za cenné pripomienky a za obetavosť počas
  tvorby mojej diplomovej práce.
} 
\veduci{akad. František Tarabák, DrSc.}
\konzultant{akad. Františka Tarabáková, DrSc.}
\pracovisko{Ústav informatiky}
\pdfzadanie{vzorove-zadanie-z-ais.pdf}

\abstract{
Donec dolor arcu, posuere at, vehicula vitae, accumsan ut,
lacus. Nulla tristique eros eu diam. Vivamus nec tortor vel
ligula elementum lacinia. Curabitur euismod eros adipiscing
ipsum. Donec sed quam at felis suscipit egestas. Morbi faucibus
libero sit amet libero.
}

\keywords{lorem ipsum, dolor, sit amet}

\abstrakt{
Abstrakt obsahuje informáciu o cieľoch práce, jej stručnom obsahu,
výsledkoch a význame celej práce. Súčasťou abstraktu je 3--5 kľúčových 
slov. Abstrakt sa píše súvisle ako jeden odsek a jeho rozsah je 
spravidla 100 až 500 slov.
}

\klucoveslova{lorem ipsum, dolor, sit amet}

\bibliographystyle{alpha}

\begin{document}
\maketitle
\newpage
\tableofcontents
\listoffigures
\listoftables
% zoznam značiek a skratiek. Pre tento koncept neexistuje
% LaTeXovsky príkaz

\uvod

Lorem ipsum dolor sit amet, consectetuer adipiscing elit.
Integer lacinia, nulla porta varius tempus, lacus metus blandit
lorem, a rutrum justo wisi id sapien. Integer risus libero,
feugiat eleifend, ornare ac, volutpat nec, sem. In facilisis,
quam eu elementum aliquet, lorem quam euismod dui, aliquet
laoreet purus ipsum ac quam. 


\chapter{Tradičný Lipsum}

Lorem ipsum dolor sit amet, consectetuer adipiscing elit.
Integer lacinia, nulla porta varius tempus, lacus metus blandit
lorem, a rutrum justo wisi id sapien. Integer risus libero,
feugiat eleifend, ornare ac, volutpat nec, sem. In facilisis,
quam eu elementum aliquet, lorem quam euismod dui, aliquet
laoreet purus ipsum ac quam. 
%
$$\int\!\!\!\int x^y\,{\rm d}x\,{\rm d}y = \int\!\!\!\int (1 + x) - (\sin^2 x + \cos^2 x)^y\,{\rm d}x\,{\rm d}y$$

Donec dolor arcu, posuere at, vehicula vitae, accumsan ut,
lacus. Nulla tristique eros eu diam. Vivamus nec tortor vel
ligula elementum lacinia. Curabitur euismod eros adipiscing
ipsum. Donec sed quam at felis suscipit egestas. Morbi faucibus
libero sit amet libero. Nullam laoreet ipsum eu eros. Donec in
diam. Ut facilisis eros vel leo. Nunc vitae mauris. Donec leo
erat, luctus porttitor, laoreet eget, facilisis non, erat.
Integer nec elit.

\section{Dolor sit amet}

Turabitur condimentum. Quisque ut risus. Vestibulum non arcu a
est feugiat porttitor. Lorem ipsum dolor sit amet, consectetuer
adipiscing elit. Duis in metus. 
%
$$\forall \epsilon>0 \exists \delta>0:\forall x\in(-\delta,\delta):|f(x)-\epsilon|<\delta$$
%
Integer ligula sapien, rutrum et, pulvinar ac, viverra a,
neque. Suspendisse hendrerit lectus id ante. Fusce mattis nunc
non ipsum. Praesent tristique hendrerit lorem. Morbi risus
erat, euismod quis.

\section{Sit transit gloria mundi}
\begin{df}
\pojem{Definíciou} nazývame každý pojem uzavretý v prostredí 
\texttt{df}.
\end{df}

Autorom predchádzajúcej definície je \osoba{František Taraba},
ktorý už v roku fusce elit enim, commodo eget, blandit eu,
faucibus sed, nisl. Maecenas adipiscing. Proin non risus in
erat dapibus hendrerit. Sed nonummy, velit sit amet dictum
eleifend, arcu lacus molestie mauris, a sollicitudin felis
lacus eget velit. Vivamus imperdiet. Vivamus accumsan
sollicitudin leo. Morbi et orci. Sed at sem. Vestibulum mattis
augue. Nam sit amet wisi. Donec vel est. Praesent vehicula.
Nunc convallis hendrerit nisl.
%
$$m^2=n_1^2+n_2^2$$

\begin{veta}[o ľuďoch]
Náhodná veličina $X$ majúca geometrické rozdelenie s parametrom $p$ vyjadruje
počet \uv{neúspechov} pred prvým úspechom pri neobmedzenej realizácii pokusov
v Bernoulliho schéme.
\end{veta}
%
\begin{dokaz}
Dôkaz prenechávame na pozorného čitateľa.
\end{dokaz}
%
\begin{dokaz}
A teraz:
$$\pmatrix{3 & 3\cr 1 & 2} \cdot 3$$
\end{dokaz}
Dôkaz prenechávame na pozorného čitateľa. Už \cite{knuth84} ukázal 
použitie tejto metódy. Na druhej strane, \cite{lamport86} použil
alternatívny prístup.\footnote{Donec venenatis rutrum odio. Fusce porta. Curabitur
a ipsum sit amet arcu semper posuere. Donec orci felis, auctor sit
amet, semper non, suscipit vel, tellus. Sed eu ipsum nec mi egestas
ultrices. Sed mauris. Aliquam purus.}

Suspendisse lobortis. Donec ornare elit sit amet nibh. Mauris nec
augue. Sed dignissim dictum mauris. Morbi tincidunt leo at est.
Pellentesque habitant morbi tristique senectus et netus et
malesuada fames ac turpis egestas. Etiam eu ante in sem dictum
eleifend. Nunc eget odio. Sed vitae elit at justo tristique
dapibus. Quisque sit amet urna at velit faucibus cursus,
ako bolo ukázané v \cite{1}.

Aliquam commodo wisi sed ipsum. Donec quis ligula. Ut porttitor,
nibh nec interdum fringilla, risus est nonummy nibh, at rutrum
massa odio molestie dolor. Morbi metus. Nulla nec velit vitae elit
nonummy lobortis. Phasellus facilisis, urna ac viverra tristique,
tellus turpis commodo dui, id pharetra erat nibh et mi. Mauris
iaculis nisl sit amet tellus molestie fringilla. 

\obrazok{lema1}{Curabitur arcu
metus, convallis eu, euismod in, ultricies nec, felis. }

Duis ligula lectus, condimentum in, lacinia eget, pellentesque sit
amet, mi. Aliquam convallis euismod arcu. Aliquam erat volutpat.
Fusce erat elit, congue eu, pretium in, congue non, neque.
Vestibulum non massa eu nisl condimentum blandit. Integer pretium
wisi id metus. Donec venenatis rutrum odio. Fusce porta. Curabitur
a ipsum sit amet arcu semper posuere. Donec orci felis, auctor sit
amet, semper non, suscipit vel, tellus. Sed eu ipsum nec mi egestas
ultrices. Sed mauris. Aliquam purus. In hac habitasse platea
dictumst. Phasellus ut urna. Etiam sit amet ligula ultrices massa
iaculis suscipit. Integer odio lacus, interdum eget, tempor eu,
aliquam nec, elit.

\begin{table}
\begin{center}
\begin{tabular}{ccc}
Level & Requirement\\
1 & 0\\
2 & 1000\\
3 & 3000\\
4 & 6000
\end{tabular}
\end{center}
\caption{Requirements of XP for given level.}
\end{table}
Suspendisse lobortis. Donec ornare elit sit amet nibh. Mauris nec
augue. Sed dignissim dictum mauris. Morbi tincidunt leo at est.
Pellentesque habitant morbi tristique senectus et netus et
malesuada fames ac turpis egestas. Etiam eu ante in sem dictum
eleifend. Nunc eget odio. Sed vitae elit at justo tristique
dapibus. Quisque sit amet urna at velit faucibus cursus.
Morbi metus. Nulla nec velit vitae elit
nonummy lobortis. Phasellus facilisis, urna ac viverra tristique,
tellus turpis commodo dui, id pharetra erat nibh et mi. Mauris
iaculis nisl sit amet tellus molestie fringilla. Sed nonummy mollis
dui. Sed fermentum suscipit metus. Curabitur quis wisi. In
condimentum pellentesque ante. Integer in odio. Curabitur arcu
metus, convallis eu, euismod in, ultricies nec, felis. Suspendisse
hendrerit ipsum ac neque pretium consectetuer. Vestibulum et diam
vitae nulla faucibus mattis. 


\chapter{Pseudonáhodné texty v súčasnosti}

Nullam neque felis, accumsan a cursus at, fringilla a massa. Praesent suscipit purus eget enim volutpat, bibendum imperdiet odio convallis. Vivamus congue ultricies dapibus. Sed eu pretium mauris, consequat scelerisque ex. Vestibulum ut sem venenatis, pharetra orci eu, tempus metus. Ut auctor porta tincidunt. Interdum et malesuada fames ac ante ipsum primis in faucibus. Sed eget risus fringilla, volutpat mauris non, cursus ex. Donec faucibus ante et nisl sodales, vel venenatis turpis rhoncus. Nunc eget congue elit. In placerat leo in libero hendrerit maximus. Vestibulum efficitur aliquet sapien blandit semper. Nulla est diam, fermentum a commodo id, ullamcorper vitae tellus. Aenean scelerisque fringilla velit id condimentum. Ut vel libero turpis.

\begin{equation}
P(X_n = x_n | X_{n - 1} = x_{n - 1}, X_{n - 2} = x_{n - 2}, \dots, X_0 = x_0) = P(X_n = x_n | X_{n - 1} = x_{n - 1})
\end{equation}

Sed ornare ac mauris semper efficitur. Aenean id sollicitudin augue. Quisque at efficitur libero, at scelerisque arcu. Suspendisse consectetur luctus posuere. Cras vel nisl non neque feugiat tempor. Aliquam dapibus, mi eget fringilla rutrum, ligula nulla mattis sem, in pharetra augue neque quis ipsum. Cras bibendum quis lacus eget interdum. Pellentesque facilisis auctor magna, vitae mattis metus fringilla in. Ut tempor vel tellus sed ultricies. Donec accumsan mi ac fermentum cursus. Donec congue, arcu et scelerisque ornare, velit leo placerat augue, in laoreet velit odio quis odio. Proin tempor gravida mauris. Nulla facilisi. Nam pellentesque porttitor enim non malesuada.

Nullam lacus turpis, tempus et nisi non, dignissim tincidunt urna. Praesent vehicula dui urna, ac ullamcorper justo laoreet in. Pellentesque at arcu id risus ultricies consequat ut in dolor. Duis posuere, enim at ullamcorper facilisis, dolor erat auctor augue, vitae congue ipsum velit vitae nibh. Maecenas tincidunt magna luctus felis tempus blandit. Pellentesque mattis lorem lacus, pretium egestas tellus maximus in. Donec ligula felis, laoreet non tortor id, ultrices lobortis arcu. Proin in fermentum libero, vitae ultricies nibh. In non euismod sapien, ullamcorper luctus tellus. Donec nibh massa, malesuada a metus ac, molestie consectetur purus. Morbi elementum molestie libero, quis condimentum ipsum.

Ut lobortis semper risus, non condimentum dui convallis ut. Nulla eget volutpat tellus. Vestibulum lobortis tincidunt massa eu rhoncus. Suspendisse luctus eu dui non vehicula. Vivamus elementum auctor felis, placerat maximus magna lobortis ut. Donec placerat sem a mi sagittis blandit. Maecenas pellentesque laoreet mauris, dictum viverra ante sodales sed. Nullam non ligula quis ante ultricies finibus non quis ex. Ut tempor vitae ipsum sed imperdiet. Fusce aliquam nisl sit amet nunc tempor convallis. Vivamus vehicula magna sit amet purus commodo, et tincidunt purus accumsan. Nulla velit dolor, lacinia nec scelerisque a, euismod a sapien. 

\zaver

Ut lobortis semper risus, non condimentum dui convallis ut. Nulla eget volutpat tellus. Vestibulum lobortis tincidunt massa eu rhoncus. Suspendisse luctus eu dui non vehicula. Vivamus elementum auctor felis, placerat maximus magna lobortis ut. Donec placerat sem a mi sagittis blandit. Maecenas pellentesque laoreet mauris, dictum viverra ante sodales sed. Nullam non ligula quis ante ultricies finibus non quis ex. Ut tempor vitae ipsum sed imperdiet. Fusce aliquam nisl sit amet nunc tempor convallis. Vivamus vehicula magna sit amet purus commodo, et tincidunt purus accumsan. Nulla velit dolor, lacinia nec scelerisque a, euismod a sapien. 
%

%\renewcommand{\bibname}{Zoznam použitej literatúry}
\begin{thebibliography}{9}
% Príklady popisu dokumentov citácií podľa systému meno a dátum (Harvardský systém)
% ----
% Varianty zápisov autorov:
%	[1] GUZANIN, Štefan, Robert SABOVČÍK a Pavol KAČMÁR. Priezviská vždy VEĽKÝMI PÍSMENAMI,
%		priezvisko prvého autora je vždy pred menom, druhý a ďalší autor majú zápis
%		Meno PRIEZVISKO
%	[2] Neuvádzať rodné mená autorov.
%	[3] Verzálky nie sú povinné, možno použiť aj iné indikatívnejšie označenie
%
% ---	
% 1. Knižná publikácia (monografia, učebnica, zborník ...)
%   1 autor
\bibitem{2}
	\osoba{Beck, G.}, 2007. \emph{Zakázaná rétorika: 30 manipulatívních technik}. Preklad
\osoba{Pomikálová, M.}. Praha: Grada Publishing. ISBN 978-80-247-1743-2.
\bibitem{3}
	\osoba{Vojčík, P.}, 2010. \emph{Občianske právo hmotné II.} 3. prep. a dopl. vyd. Košice: UPJŠ v Košiciach. ISBN 978-80-7097-817-7.
%  2 autori
\bibitem{4}
	\osoba{Šoltés, M.} a \osoba{Radoňák, J.}, 2013. \emph{Základné princípy laparoskopickej chirurgie.} Košice: UPJŠ v Košiciach. ISBN 978-80-8152-074-7.
% 3 autori
\bibitem{5}
	\osoba{Guzanin, Š.}, \osoba{Sabovčík, R.} a \osoba{Kačmár, P.}, 2004. \emph{Selected Chapters of Plastic and Reconstructive Surgery: vysokoškolské učebné texty}. Košice: Univerzita Pavla Jozefa Šafárika v Košiciach, Lekárska fakulta. ISBN 80-7097-557-1.
% 4 a viac autorov
\bibitem{6}
	\osoba{Nagyová, I.} et al. 2009. \emph{Measuring health and quality of life in the chronically ill}. Košice: Equilibria. ISBN 978-80-892-8446-7.
% Elektronická kniha
\bibitem{7}
	\osoba{Speight, J. G.}, 2005. \emph{Lange's Handbook of Chemistry} [online]. London: McGraw-Hill. [cit. 2009.06.10.] ISBN 978-1-60119-261-5. Dostupné na: \url{http://www.knovel.com/web/portal/basic_search/display?_EXT_KNOVEL_DISPLAY_bookid=1347&_EXT_KNOVEL_DISPLAY_fromSearch=true&_EXT_KNOVEL_DISPLAY_searchType=basic}
% zborník
\bibitem{8}
	\osoba{Bačkor, M.} a \osoba{Mihaličová, S.}, zost., 2013. \emph{Zborník príspevkov z konferencie 11. dni doktorandov experimentálnej biológie rastlín a 13. konferencie experimentálnej biológie rastlín} [online]. Košice: Univerzita Pavla Jozefa Šafárika v Košiciach, Prírodovedecká fakulta [cit. 2009-06-10]. ISBN 9788081520327. Dostupné na: \url{
http://www.upjs.sk/public/media/5596/PF-Zbornik-prispevkov-konferencie-11-dni-doktorandov.pdf}
%
% 2. Časopis (ako celok)
\bibitem{9}
	\emph{Thaiszia: Journal of Botany}. Košice: P.J.Safarik University, Botanic Garden, \mbox{1990--\ .} ISSN 1210-0420.
\bibitem{10}
	\emph{Ikaros: elektronický časopis o informační bezpečnosti} [online], 2002. [Praha]: Ikaros. 1997--{} [cit. 2002-03-08]. Dostupné na: \url{http://www.ikaros.cz/}. ISSN 1212-5075.
% Jedno číslo časopisu
\bibitem{11}
	\emph{CHIP: magazín informačních technologií}, 2013. Praha: Burda Praha, roč. 23, říjen. ISSN 1210-0684.
%  ------
%  3. Príspevok v knihe/zborníku
%  ------
\bibitem{12}
	\osoba{Sabol, J.}, 2000. Jazyk ako ľudské posolstvo: (namiesto doslovu). In: \emph{O jazyku a štýle kriticky aj prakticky}. Prešov: Náuka, s. 149--159. ISBN 809676022X.
\bibitem{13}
	\osoba{Tóthová, E.} a kol., 2013. A rare t(9,22,16)(q34,q11,q24) translocation in chronic myeloid leukemia for which imatinib mesylate was effective: a case report. In: \emph{XXVII. Olomoucké hematologické dny s mezinárodní účastí, 12.--14.5.2013, Olomouc: sborník abstrakt}. Olomouc: Univerzita Palackého v Olomouci, s. 75--76. ISBN 9788024434803.
%  ------
% 4. Článok v časopise
%  ------
\bibitem{14}
	\osoba{Beňačka, J.} et al., 2009. A better cosine approximate solution to pendulum equation. In: \emph{International Journal of Mathematical Education in Science and Technology}. Vol. 40, no. 2, p. 206--215. ISSN 0020-739X.
\bibitem{15} 
	\osoba{Dubayová, T.} et al., 2010. The impact of the intensity of fear on patient's delay regarding health care seeking behavior: a systematic review vyhľadaní zdravotníckej starostlivosti. In: \emph{International Journal of Public Health}. Vol. 55, no. 5, p. 459--468. ISSN 1661-8556.
\bibitem{16} 
	\osoba{Steinerová, J.}, 2000. Princípy formovania vzdelania v informačnej vede. In: \emph{Pedagogická revue}. Roč. 2, č. 3, s. 8--16. ISSN 1335-1982.
\bibitem{17}
	\osoba{Hoggan, D.}, 2002. Challenges, Strategies, and Tools for Research Scientists. In: \emph{Electronic Journal of Academic and Special Librarianship} [online]. Vol. 3, no. 3 [cit. 2013-01-10]. ISSN 1525-321X. Dostupné na: \url{http://southernlibrarianship.icaap.org/content/v03n03/Hoggan_d01.htm}
\bibitem{18}
	\osoba{Srbecká, Gabriela}, 2010. Rozvoj kompetencí studentů ve vzdělávání. In: \emph{Inflow: information journal} [online]. Roč. 3, č. 7 [cit. 2013-08-06]. ISSN 1802-9736. Dostupné na: \url{http://www.inflow.cz/rozvoj-kompetenci-studentu-ve-vzdelavani}
%  ------
% 5. Príspevok v zborníku na CD-ROM
%  ------
\bibitem{19}
	\osoba{Zemánek, P.}, 2001. The machines for ``green works'' in vineyards and their economical evaluation. In: \emph{9th International Conference: proceedings. Vol. 2. Fruit Growing and viticulture} [CD-ROM]. Lednice: Mendel University of Agriculture and Forestry, p. 262--268. ISBN 80-7157-524-0.
%  ------
% 6. Záverečné a kvalifikačné práce
%  ------
\bibitem{20} 
	\osoba{Mikulášiková, M.}, 1999. \emph{Didaktické pomôcky pre praktickú výučbu na hodinách výtvarnej výchovy pre 2. stupeň základných škôl}: diplomová práca. Nitra: UKF.
\bibitem{21} 
	\osoba{Urdzík, P.}, 2007. \emph{Predikcia intrauterinnej rastovej retardácie a preeklampsie pomocou biochemických a ultrazvukových markerov}: dizertačná práca. Košice: UPJŠ v Košiciach.
%  ------
% 7. Výskumné správy
%  ------
\bibitem{22} 
	\osoba{Baumgartner, J.} a kol., 1998. \emph{Ochrana a udržiavanie genofondu zvierat, šľachtenie zvierat}: výskumná správa. Nitra: VÚŽV.
%  ------
% 8. Normy
%  ------
\bibitem{23}
	STN ISO 690: 2012. \emph{Informácie a dokumentácia. Návod na tvorbu bibliografických odkazov na informačné pramene a ich citovanie}.
%  ------
% 9. Mapa
%  ------
\bibitem{24} 
	VKÚ, 2003. \emph{Košice: mapa okolia}. [1:15000]. 3. vyd. Harmanec: VKÚ. ISBN 80-8042-223-0.
%  ------
% 10. Zákon
%  ------
\bibitem{25} 
	\emph{Zákon č.131/2002 Zb. o vysokých školách a o zmene a doplnení niektorých zákonov.}
\bibitem{26} 
	\emph{Zákon č. 313/2001 o verejnej službe.}

\end{thebibliography}
%
%
\prilohy
\priloha{ASCII Art}
\begin{verbatim}
 ____       _ _       _                _    
|  _ \ _ __(_) | ___ | |__   __ _     / \   
| |_) | '__| | |/ _ \| '_ \ / _` |   / _ \  
|  __/| |  | | | (_) | | | | (_| |  / ___ \ 
|_|   |_|  |_|_|\___/|_| |_|\__,_| /_/   \_\
                                             
\end{verbatim}

\priloha{Druhý ASCII Art}
\begin{verbatim}
 ____       _ _       _             ____  
|  _ \ _ __(_) | ___ | |__   __ _  | __ ) 
| |_) | '__| | |/ _ \| '_ \ / _` | |  _ \ 
|  __/| |  | | | (_) | | | | (_| | | |_) |
|_|   |_|  |_|_|\___/|_| |_|\__,_| |____/ 

\end{verbatim}


\end{document}