\documentclass[thesismargins]{rnthesis}
\usepackage[slovak]{babel}
\usepackage[T1]{fontenc}
\usepackage[utf8]{inputenc}
\usepackage{lmodern}
\usepackage{rnt-thm}

\usepackage[numbers]{natbib}
\renewcommand{\bibnumfmt}[1]{\hspace{-6pt}}

\title{O dizertačných prácach}
\typprace{dizertačná}
\author{Bc. Milan Makrela}
\pracovisko{Ústav informatiky}
\veduci{doc. RNDr. Jozef Mak, DrSc.}
\miesto{Košice}
\rok{2019}

\abstract{
Donec dolor arcu, posuere at, vehicula vitae, accumsan ut,
lacus. Nulla tristique eros eu diam. Vivamus nec tortor vel
ligula elementum lacinia. Curabitur euismod eros adipiscing
ipsum. Donec sed quam at felis suscipit egestas. Morbi faucibus
libero sit amet libero.
}

\abstrakt{
Nulla tristique eros eu diam. Vivamus nec tortor vel
ligula elementum lacinia. Curabitur euismod eros adipiscing
ipsum. Donec sed quam at felis suscipit egestas. Morbi faucibus
libero sit amet libero.
}

\begin{document}
\maketitle
\newpage
\chapter{Tradičný Lipsum}
Lorem ipsum dolor sit amet, set \cite{beck07} et \cite{vojcik10}.

\begin{thebibliography}{9}
% Príklady popisu dokumentov citácií podľa systému meno a dátum (Harvardský systém)
% ----
% Varianty zápisov autorov:
%	[1] GUZANIN, Štefan, Robert SABOVČÍK a Pavol KAČMÁR. Priezviská vždy VEĽKÝMI PÍSMENAMI,
%		priezvisko prvého autora je vždy pred menom, druhý a ďalší autor majú zápis
%		Meno PRIEZVISKO
%	[2] Neuvádzať rodné mená autorov.
%	[3] Verzálky nie sú povinné, možno použiť aj iné indikatívnejšie označenie
%
% ---	
% 1. Knižná publikácia (monografia, učebnica, zborník ...)
%   1 autor
\bibitem{beck07}
	\osoba{Beck, G.}, 2007. \emph{Zakázaná rétorika: 30 manipulatívních technik}. Preklad
\osoba{Pomikálová, M.}. Praha: Grada Publishing. ISBN 978-80-247-1743-2.
\bibitem{vojcik10}
	\osoba{Vojčík, P.}, 2010. \emph{Občianske právo hmotné II.} 3. prep. a dopl. vyd. Košice: UPJŠ v Košiciach. ISBN 978-80-7097-817-7.
\end{thebibliography}
\end{document}
