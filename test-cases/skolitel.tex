\documentclass{rnthesis}
\usepackage[slovak]{babel}
\usepackage[T1]{fontenc}
\usepackage[utf8]{inputenc}
\usepackage{lmodern}
\usepackage{rnt-thm}


\title{Školiteľ}
\author{Bc. Milan Makrela}
\pracovisko{Ústav informatiky}
\typprace{diplomová}
\skolitel{doc. RNDr. Jozef Mak, DrSc.}
\miesto{Košice}
\rok{2017}

\begin{document}
\maketitle
\newpage
\chapter{Tradičný Lipsum}
Lorem ipsum dolor sit amet.

\begin{equation}
P(X_n = x_n | X_{n - 1} = x_{n - 1}, X_{n - 2} = x_{n - 2}, \dots, X_0 = x_0) = P(X_n = x_n | X_{n - 1} = x_{n - 1})
\end{equation}

Iný príklad:

\begin{equation}
\int\!\!\!\int x^y\,{\rm d}x\,{\rm d}y = \int\!\!\!\int (1 + x) - (\sin^2 x + \cos^2 x)^y\,{\rm d}x\,{\rm d}y
\end{equation}

\begin{veta}[o ľuďoch]
Náhodná veličina $X$ majúca geometrické rozdelenie s parametrom $p$ vyjadruje
počet \uv{neúspechov} pred prvým úspechom pri neobmedzenej realizácii pokusov
v Bernoulliho schéme.
\end{veta}
%
\begin{dokaz}
Dôkaz prenechávame na pozorného čitateľa.
\end{dokaz}


%---------

\chapter{Dolor sit amet}
Potom môžeme usúdiť, že 

\begin{equation}
\pmatrix{3 & 3\cr 1 & 2} \cdot 3
\end{equation}

\begin{veta}[tá istá ľuďoch]
Náhodná veličina $X$ majúca geometrické rozdelenie s parametrom $p$ vyjadruje
počet \uv{neúspechov} pred prvým úspechom pri neobmedzenej realizácii pokusov
v Bernoulliho schéme.
\end{veta}



\end{document}
